\section{Questão 1} 
A melhoria dos processos de desenvolvimento de software pode ajudar a diminuir o impacto de algumas das dificuldades essenciais descritas por Brooks? E as acidentais?

\section{Questão 2}

O processo de conserto de maquinas industriais é dado da seguinte forma:
\begin{itemize}
    \item Entrada

        Peça defeituosa

    \item Saída

        Nova peça.

    \item Papeis
        \begin{itemize}
            \item Técnico: responsável por levantar os requisitos da peca.
            \item Engenheiro: responsável por projetar a nova peca.
        \end{itemize}
        
    \item Atividades
       \begin{itemize}
        \item Levantar medidas da peca original.
        \item Projetar nova peça.
        \item Manufaturar a peça.
        \item Repor a peça.
       \end{itemize} 
    \item Ferramentas
       Paquímetro, computador, software Cad.
    
\end{itemize}

\section{Questão 3}
    Empresas de tecnologia conhecidas como startups, via de regra, utilizam
processos imaturos devido a seu pouco tempo de existência. Essas empresas
focam na construção de um produto e somente quanto mais experientes podem
investir na melhoria dos seus processos interno. Por outro lado, bancos
são instituições bastante antigas que tem processos bem definidos. Um cliente
chega ao banco e require uma senha e só depois é atendido, é um exemplo de 
processo bem definido e maduro. 

\section{Questão 4}
\begin{itemize}
    \item Iniciação

    Selecionar contexto é a atividade responsável por escolher dentro de uma organização a área que se deseja atuar para melhorar.

    \item Diagnóstico

    Desenvolver recomendações tem como finalidade desenvolver a partir da experiências e literatura possíveis melhorias para o processo.

    \item Planejamento


    Desenvolver plano de implantação visa descrever como as modificações escolhidas serão implementadas na organização.

    \item Ação

    Criar solução tem por objetivo desenvolver a solução para o problema diagnosticado.

    \item Aprendizado

    Propor ações futuras caracteriza-se por ser uma atividade que auxilie a organização em definir os próximos projetos de MPS.

\end{itemize}

\section{Questão 5} 
\begin{itemize}
    \item Planejar e preparar avaliação

    Selecionar e preparar equipe visa escolher os indivíduos e treiná-los para desempenhar as atividade de avaliação.

    \item Conduzir a avaliação

    Gerar resultados da avaliação cria documento que abrange o que foi encontrado durante a avaliação.

    \item Relatar resultados

    Entregar resultados da avaliação é a atividade responsável por fazer os resultados chegar aos interessados no projeto de MPS.
 
\end{itemize}

    
