\documentclass{article}
\title{Resumo artigo}
\date{2013-09-01}
\author{Gustavo Coelho}
\usepackage[utf8]{inputenc}
\usepackage{graphicx}
\usepackage{setspace}
\usepackage[brazil]{babel}
\graphicspath{ {images/} }
\usepackage{geometry}
 \geometry{
 a4paper,
 total={170mm,257mm},
 left=30mm,
 top=30mm,
 right=20mm,
 bottom=20mm,
 }
\begin{document}
\newcommand{\myid}{Gustavo Rodrigues Coelho, 11/0030559 }
\newcommand{\teacher}{André Barros}
\newcommand{\discipline}{IHC}
\newcommand{\mainTitle}{SWEBOK}
\newcommand{\city}{Brasília}
\newcommand{\university}{Universidade de Brasília}
\newcommand{\faculty}{Faculdade Unb Gama}


\begin{titlepage}
        \begin{center}
            \includegraphics[height=3cm]{logo.jpg}\\[0.3cm]
            {\large \university}\\[0.2cm]
            {\large \faculty}\\[0.2cm]
            {\large \discipline}\\[4.1cm]
            {\bf \huge \mainTitle}\\[4.1cm]
        \end{center}
        \raggedleft{\normalsize \myid}\\[0.7cm]
        {\normalsize Professor(a): \teacher}
        \vfill
        \begin{center}
            {\large \city}\\[0.2cm]
            {\large \the\year}
        \end{center}
\end{titlepage}


\tableofcontents{}
\newpage

\section{Introdução}
\subsection{Visão Geral}
 O plano de Gerência de Configuração de Software(GCS) será o guia para se implementar as necessidade de GCS do projeto CodeSchool. O projeto esta sendo desenvolvido para auxiliar os alunos no processo de aprendizagem da disciplina de computação básica. O projeto visa construir uma plataforma web onde alunos possam interagir com colegas e professores facilitando seu desenvolvimento durante o período do curso. O projeto conta com alguns modulos que podem ser vistos abaixo: 

\begin{itemize}
    \item[] cs\_activities: Reune as atividades propostas pelo professor do curso.
    \item[] cs\_core: Inclui as funcionalidades compartilhadas entre os módulos. 
    \item[] cs\_linktable:  
    \item[] cs\_questions: Gerencia e exibição das questões. 
    \item[] cs\_auth: Cuida da autenticacão, cadastro e manutencão dos usuários. 
    \item[] cs\_courses: Gerencia os cursos. 
    \item[] cs\_pages: Eu nao sei ainda 
    \item[] cs\_search: Eu nao sei 
\end{itemize}

O software CodeSchool pode ser encontrado no link à seguir:\\

https://github.com/fabiommendes/codeschool.git \\

O projeto se encontra em fase inicial de desenvolvimento não contando com muitos processos automatizados. No cenário atual o usuário deve baixar o código do repositório remoto e instalar as dependências utilizando o gerenciador de pacotes pip. Porém, são necessárias outras dependências que devem ser instaladas manualmente como o send\_box. O objetivo desse documento é levantar os principais problemas para criação do ambiente de desenvolvimento e estrátegias para solucionar esses problemas.

\subsection{Próposito}
Esse documento tem por objetivo propor um projeto para aplicação de práticas e e conceitos de gerência de configuração ao software no software CodeSchool. Esse projeto será desenvolvido durante a disciplina de GCS e tem com principal finalidade contribuir para a melhoria dos processos de contribuicao do softare CodeSchool. Com isso visamos facilitar a criacão de ambientes de desenvolvimento em ambientes Linux. 
\subsection{Escopo}
\newpage


\end{document}
