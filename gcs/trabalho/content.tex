\documentclass{article}
\title{Resumo artigo}
\date{2013-09-01}
\author{Gustavo Coelho}
\usepackage[utf8]{inputenc}
\usepackage{graphicx}
\usepackage{setspace}
\usepackage[brazil]{babel}
\graphicspath{ {images/} }
\usepackage{geometry}
 \geometry{
 a4paper,
 total={170mm,257mm},
 left=30mm,
 top=30mm,
 right=20mm,
 bottom=20mm,
 }
\begin{document}
\newcommand{\myid}{Gustavo Rodrigues Coelho, 11/0030559 }
\newcommand{\teacher}{André Barros}
\newcommand{\discipline}{IHC}
\newcommand{\mainTitle}{SWEBOK}
\newcommand{\city}{Brasília}
\newcommand{\university}{Universidade de Brasília}
\newcommand{\faculty}{Faculdade Unb Gama}


\begin{titlepage}
        \begin{center}
            \includegraphics[height=3cm]{logo.jpg}\\[0.3cm]
            {\large \university}\\[0.2cm]
            {\large \faculty}\\[0.2cm]
            {\large \discipline}\\[4.1cm]
            {\bf \huge \mainTitle}\\[4.1cm]
        \end{center}
        \raggedleft{\normalsize \myid}\\[0.7cm]
        {\normalsize Professor(a): \teacher}
        \vfill
        \begin{center}
            {\large \city}\\[0.2cm]
            {\large \the\year}
        \end{center}
\end{titlepage}


\tableofcontents{}
\newpage

\section{Introdução}
\subsection{Visão Geral}
 O plano de Gerência de Configuração de Software(GCS) será o guia para se implementar as necessidade de GCS do projeto CodeSchool. O projeto esta sendo desenvolvido para auxiliar os alunos no processo de aprendizagem da disciplina de computação básica. O projeto visa construir uma plataforma web onde alunos possam interagir com colegas e professores facilitando seu desenvolvimento durante o período do curso. O projeto conta com alguns modulos que podem ser vistos abaixo: 

\begin{itemize}
    \item[] cs\_activities: Reune as atividades propostas pelo professor do curso.
    \item[] cs\_core: Inclui as funcionalidades compartilhadas entre os módulos. 
    \item[] cs\_questions: Gerencia e exibição das questões. 
    \item[] cs\_auth: Cuida da autenticacão, cadastro e manutencão dos usuários. 
    \item[] cs\_courses: Gerencia os cursos. 
\end{itemize}

O software CodeSchool pode ser encontrado no link à seguir:\\

https://github.com/fabiommendes/codeschool.git \\

O projeto se encontra em fase inicial de desenvolvimento não contando com muitos processos automatizados. No cenário atual o usuário deve baixar o código do repositório remoto e instalar as dependências utilizando o gerenciador de pacotes pip. Porém, são necessárias outras dependências que devem ser instaladas manualmente como o send\_box. O objetivo desse documento é levantar os principais problemas para criação do ambiente de desenvolvimento e estrátegias para solucionar esses problemas.

\subsection{Próposito}

Esse documento tem por objetivo propor um projeto para aplicação de práticas e e conceitos de gerência de configuração ao software para o (falta algo aqui) CodeSchool. Esse projeto será desenvolvido durante a disciplina de GCS e tem com principal finalidade contribuir para a melhoria dos processos de desenvolvimento colaborativo do softare CodeSchool. Com isso visamos facilitar a criacão de ambientes de desenvolvimento em diferentes ambientes. 

\subsection{Escopo}

O software CodeSchool é utilizado como case na disciplian de Programacão Web, portanto, serão levantados requisitos junto a essa disciplina para o desenvolvimento desse projeto. Entre as demandas destacadas para atender e facilitar o aprendizado dos alunos da disciplina será disponibilizada formas automatizadas de criacão do ambiente de desenvolvimento. As principais formas levantadas até o momento e que não se limitam as apresentadas serão:

\begin{itemize}
    \item Docker
    \item Vagrant 
    \item Chef
\end{itemize}
  
\newpage

\section{Produtos}

Durante a disciplina serão desenvolvidos os seguintes produtos para disponibilizar ambientes de desenvolvimento rápidas e compartilhadas para que o usário não se preocupe com esse ponto.

\subsection{Docker}

A virtualizacão simula um ambiente dentro do sistema operacional hospedeiro, assim isola a aplicacão do sistema operacional. A vantagem desse recurso é compartilhamento de recursos físicos entre aplicacões. Docker é uma ferramenta que utiliza essa tecnologia para criar ambientes isolados para rodar códigos, ambientes de desenvolvimento e deploy de aplicacões. Uma outra vantagem dessa plataforma é a possibilidade de compartilhar esses ambientes através de uma plataforma online. Assim, um usuário pode criar o ambiente virtual e compartilha-lo, sendo uma alteracão propagada para todos os outros usuários que possuam esse ambiente. Durante a disciplina será criado um container para o ambiente de desenvolvimento que será disponibilizado no web site: https://hub.docker.com/ que é o respositório central de ambientes. 

\subsection{Chef}

\subsection{Vagrant}


\end{document}
