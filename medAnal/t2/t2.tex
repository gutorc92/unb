\documentclass{article}
\title{Resumo artigo}
\date{2013-09-01}
\author{Gustavo Coelho}
\usepackage[utf8]{inputenc}
\usepackage{graphicx}
\usepackage{setspace}
\usepackage[brazil]{babel}
\graphicspath{ {images/} }
\usepackage{geometry}
 \geometry{
 a4paper,
 total={170mm,257mm},
 left=30mm,
 top=30mm,
 right=20mm,
 bottom=20mm,
 }
\begin{document}
\begin{titlepage}
        \begin{center}
            \includegraphics[height=3cm]{logo.jpg}\\[0.3cm]
            {\large Universidade de Brasília}\\[0.2cm]
            {\large Faculdade Unb Gama}\\[0.2cm]
            {\large Medição e Análise}\\[4.1cm]
            {\bf \huge Software Metrics - Seção 3.1}\\[4.1cm]
        \end{center}
        \raggedleft{\normalsize Gustavo Rodrigues Coelho, 11/0030559}\\[0.7cm]
        {\normalsize Professor(a): Elaine Venson}
        \vfill
        \begin{center}
            {\large Brasília}\\[0.2cm]
            {\large 2016}
        \end{center}
\end{titlepage}



\newpage
A seção "A goal-based framework for software measurement" contextualiza preocupações inerentes ao processo de medição de software. Primeiramente o autor aponta quais são as entidade mensurávies na indústria de software. Os três mais comuns são:
\begin{itemize}
\item    Processos
        \begin{itemize}
        \item[] Refere-se a como é construido software.
        \end{itemize}
\item    Produtos
        \begin{itemize}
        \item[] O que é produzido ao final do processo.
        \end{itemize}
\item    Recursos
        \begin{itemize}
        \item[] Materiais utilizados nos processos e para a construção dos produtos.
        \end{itemize}
\end{itemize}

O artigo ainda aponta que as entidades podem ser mensurados em dois tipos de aspectos e as suas diferenças. Os dois aspectos descritos podem assim ser definidos:
\begin{itemize}
\item   Internos
        \begin{itemize}
        \item[] Os atributos internos referem-se a própria entidade.
        \end{itemize}
\item    Externos 
        \begin{itemize}
        \item[]  Verifica-se o impacto da entidade em um ambiente.
        \end{itemize}
\end{itemize}

As caractéristicas dos aspectos internos permitem o maior controle de seus atributos tornando o processo de medição mais fácil para as organizações. Devido a isso, os aspectos internos são mensurado primeiramente do que os externos. Porém, as empresas de software são avalidas por aspectos externos - como falhas, confiabilidade, etc - o que levou as pesquisas na área de medição e análise a procurar como os aspectos internos afetam os aspectos externos.  

O próximo passo do artigo é identificar quais os aspectos de cada entidade que podem ser mensurados. O primeiro a ser descrito são os processos.

\begin{itemize}
\item Processos
    \begin{itemize}
        \item[] Os processos tem como seus principais atributos a duração(tempos gasto para execução), o esforço(pessoas utilizadas) e os incidentes.  
    \end{itemize}

\item Produtos
    \begin{itemize}
        \item[] O objetivo principal da indústria de software é produzir software, sendo esse o principal produto. Por isso essa categoria tem dois tipos de atributos
    \begin{itemize}
        \item Externos
            \begin{itemize}
            \item[] Usabilidade e integridade.
            \end{itemize}
        \item Internos
            \begin{itemize}
            \item[] Tamanho e esforço.
            \end{itemize}
    \end{itemize}
    \end{itemize}


\item Recursos
    \begin{itemize}
        \item[] Pessoas 
            \begin{itemize}
                \item[] As pessoas são o principal insumo na indústria de software sendo portanto um grande foco de avaliação dentro da organização. Um aspecto a ser considerado pela equipe de medição das empresas é que diferente de outros processos a quantidade de software produzida pelo número de pessoas utilizadas pode não ser uma boa medida de produtividade. O autor recomenda a utilização de outros fatores para mensurar a produtividade da equipe. 
            \end{itemize}
        \item[] Materiais
        \item[] Ferramentas
    \end{itemize}
\end{itemize} 

O artigo finaliza descrevendo as dificuldades em encontrar medidadas básicas para compor medidas pois a medida composta tem que refletir aspectos relevantes para a organização. 



\end{document}
