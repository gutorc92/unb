\documentclass{article}
\title{Resumo artigo}
\date{2013-09-01}
\author{Gustavo Coelho}
\usepackage[utf8]{inputenc}
\usepackage{graphicx}
\usepackage{setspace}
\usepackage[brazil]{babel}
\graphicspath{ {images/} }
\usepackage{geometry}
 \geometry{
 a4paper,
 total={170mm,257mm},
 left=30mm,
 top=30mm,
 right=20mm,
 bottom=20mm,
 }
\begin{document}
\newcommand{\myid}{Gustavo Rodrigues Coelho, 11/0030559 }
\newcommand{\teacher}{André Barros}
\newcommand{\discipline}{IHC}
\newcommand{\mainTitle}{SWEBOK}
\newcommand{\city}{Brasília}
\newcommand{\university}{Universidade de Brasília}
\newcommand{\faculty}{Faculdade Unb Gama}


\begin{titlepage}
        \begin{center}
            \includegraphics[height=3cm]{logo.jpg}\\[0.3cm]
            {\large \university}\\[0.2cm]
            {\large \faculty}\\[0.2cm]
            {\large \discipline}\\[4.1cm]
            {\bf \huge \mainTitle}\\[4.1cm]
        \end{center}
        \raggedleft{\normalsize \myid}\\[0.7cm]
        {\normalsize Professor(a): \teacher}
        \vfill
        \begin{center}
            {\large \city}\\[0.2cm]
            {\large \the\year}
        \end{center}
\end{titlepage}



\newpage
A indústria de software tem buscado o desenvolvimento de padrões de qualidade para a construção de 
produtos que satisfação as necessidade dos usuários. O artigo apresenta em sua introdução a importância
da qualidade para o desenvolvimento de software. A falta de parâmetros de qualidade pode estancar o 
desenvolvimento de um projeto tornando-o sem valor para a organização. 

A preocupação com o atendimento das necessidades do usuário fez com que ISO desenvolvesse  uma definição
de parâmetros para avaliar a qualidade dos projetos. Históricamente essa tarefa pode ser mosntrada
pela história das normas criadas pela entidade. A série 25000 das normas conhecidas como SQuaRe foram 
a base para a definição de qualidade para a área de software e recebeu diversas revisões para se
adequar as novas realidades de mercado.

A série 25000 divide a qualidade em 5 grupos: qualidade de requisitos, qualidade de gerenciamento,
qualidade de mensuração, qualidade de evaliação e qualidade de modelo. Essas áreas abarcam os principais
pontos do desenvolvimento de software e é capaz de oferecer um arcabouço teórico para a implatação de 
processos de qualidade nas organizações.

Um processo eficaz de qualidade definido pela ISO segue duas atividades principais no tocante a qualidade.
O primeiro passo seria a definição, com subsídio da série 25000 e outras, dos parâmetros de qualidade do 
ponto de vista dos interessados no projeto. Esse passo garate a o sentido do projeto e a satisfação do cliente.
O segundo passo é definir os termos de qualidade interna que interessam a equipe. Ao final do desenvolvimento,
seria criado um documento específico para analisar os resultados interno e externo do projeto em questão.





\end{document}
