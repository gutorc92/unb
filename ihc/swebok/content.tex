\documentclass{article}
\title{Resumo artigo}
\date{2013-09-01}
\author{Gustavo Coelho}
\usepackage[utf8]{inputenc}
\usepackage{graphicx}
\usepackage{setspace}
\usepackage[brazil]{babel}
\graphicspath{ {images/} }
\usepackage{geometry}
 \geometry{
 a4paper,
 total={170mm,257mm},
 left=30mm,
 top=30mm,
 right=20mm,
 bottom=20mm,
 }
\begin{document}
\newcommand{\myid}{Gustavo Rodrigues Coelho, 11/0030559 }
\newcommand{\teacher}{André Barros}
\newcommand{\discipline}{IHC}
\newcommand{\mainTitle}{SWEBOK}
\newcommand{\city}{Brasília}
\newcommand{\university}{Universidade de Brasília}
\newcommand{\faculty}{Faculdade Unb Gama}


\begin{titlepage}
        \begin{center}
            \includegraphics[height=3cm]{logo.jpg}\\[0.3cm]
            {\large \university}\\[0.2cm]
            {\large \faculty}\\[0.2cm]
            {\large \discipline}\\[4.1cm]
            {\bf \huge \mainTitle}\\[4.1cm]
        \end{center}
        \raggedleft{\normalsize \myid}\\[0.7cm]
        {\normalsize Professor(a): \teacher}
        \vfill
        \begin{center}
            {\large \city}\\[0.2cm]
            {\large \the\year}
        \end{center}
\end{titlepage}

\tableofcontents{}

\newpage
\section{SWEBOK}

\begin{enumerate}
\item O que é o SWEBOK? Por quem e para quem ele foi escrito?  
	\begin{itemize}
	\item[] O livro Software Engineering Body of Knowledge pode ser definido como um guia para os profissionais da área de engenharia de software sobre as habilidades que serão necessárias para um bom desempenho da profissão. 
 O guia ainda pode ser usadao para a definição de cursos da área. Vale resaltar que o livro é uma reunião dos conhecimentos estabelecidos até o momento para a área de engenharia de software, portanto, pode evoluir e mudar de acordo com o desenvolvimento de novas teorias. O guia foi desenvolvido pela IEEE a mais renomada institução de engenharia pela primeira vez em 2004, a versão 3.0 é uma revisão da versão de 2004.  
	\end{itemize}
\item Baseado   no   SWEBOK   v.   3.0,   descreva   o   que   é   Projeto   de   Interface   do   Usuário   (User  
Interface Design). 
	\begin{itemize}
	\item[] Softwares são desenvolvidos para serem utilizados por seres humanos que são seres altamente sigulares.
 Para assegurar que os operadores possam utilizar da melhor forma possível essas ferramentas foi desenvolvido 
uma área dentro da engenharia de software que se preocupa com essa questão. 
O projeto de interface do Usuário visa assegurar uma experiência amigaval pra 
o usuário levando em consideração alguns princípios como: facilidade de aprendizagem, familiariedade, consitência entre outros.
O design de interface faz parte da área de design de software e tem como  suas principais questões: como o usuário
interage com o software e como o software deve interagir com o usuário. A reposta para essas questões pode usar alguns estilos de comunicação como: pergunta-resposta, seleção de menu, entre outras. Ainda é parte do design de interface
a preocupação com a forma de apresentar a informação após o processamento para o usuário e a localização geográfica do usuário. 
A finalidade dessa área é alcançada utilizando-se
um método com três fases: análise do usário, prototipação e mensurar a interface. 
	\end{itemize}
\item Baseado   no   SWEBOK   v.   3.0,   descreva   o   que   é   Teste   de   Software.   Essa   descrição  
deve conter sobre Testes de Usabilidade e Interação Humano Computador. 
	\begin{itemize}
	\item[] O teste de software visa verificar se o sotware provê as saídas esperadas para um determinado conjunto de entradas. Um software pode ser testado infinitamente dada a quantidade de várias que podem ser alteradas e o domínio delas. Assim o teste focaliza as entradas que são mais relevantes para serem testadas. Os testes podem ser dividos em levels baseados nos objetivos do teste. Teste de Usabilidade e Interação Humano Computador são dois levels de teste. O primeiro tem como objetivo verificar se os componentes de interface estão provendo o necessário fluxo de dados com o usuário. Por sua vez o teste de Interação Humano Computador, verifica a facilidade do usuário final aprender a operar o software.
	\end{itemize}
\end{enumerate}

\newpage
\section{Referência}
SWEBOK V3 pages: xvii, Cap 2 item 4, Cap 4 item 2.
\end{document}
