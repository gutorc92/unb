\setcounter{secnumdepth}{0}
\section{Resenha Temple Grandi filme}
O filme de Temple Grandi apresenta a vida de um ser humano que supera
limitações e desafios durante sua vida para construir uma brilhante carreira.
Diagnosticada com autismo ainda criança, a personagem principal é incentivada
por sua mãe a persistir e interagir com outras pessoas. Para ter uma formação
apropriada e atenção adequada Temple é enviada a uma escola integral, onde com
a ajuda de um professor desenvolve suas habilidades. O autismo dificulta 
sua interação com as pessoas porém da a personagem uma incrível habilidade de 
memorização e percepção. Essas habilidades que ela utiliza durante o filme
para desenvolver novos equipamentos para a industria agropecuária e frigorifica.
O texto irá apresentar como esses produtos se relacionam com a iteração humano
computador e como as habilidades da personagem foram importantes para o desenvolvimento
desses produtos. Além disso, será mostrado como um design de iteração pode aproveitar
algumas experiências vistas no filme.

Durante o seu tempo de escola Temple percebe que uma das limitações de seu autismo
é não aceitar o contato humano. Essa atitude causa certos embaraços a personagem
por não conseguir compreender o afeto que outros seres humanos dispensa a ela. 
Desafiada por seus mestres a pensar em algum tipo de mitigação do problema, Temple
desenvolve uma maquina que simula um abraco humano. O equipamento permite novas sensações
e a melhoria da sua interação com os outros seres humanos. Um ponto importante 
a ressaltar dessa primeira invenção da personagem é a pesquisa extensa feita para
produzir a máquina além do seu design. O usuário poderia regular a intensidade do abraco
simulando o contato humano. Muitas vezes o design de iteração humano computador
terá que pesquisar intensamente - como fez Temple -  sobre o tipo de usuário para 
qual o sistema esta sendo desenvolvido para melhor adequar o design do software ou produto.

Continuando em sua trajetória a personagem Temple estuda ciências naturais e resolve
fazer mestrado em bovinos. A personagem então tenta entender o comportamento dos bovinos
para isso ela se comporta como um: andando pelos currais, pelos corredores, dentre outras
coisas. Aqui temos uma grande lição de como, as vezes, é necessário entrar no contexto
do usuário para entender suas necessidades e o seu comportamento para desenvolver um
produto adequado. Temple, então, projeta novas formas de currais que aproveitam o 
comportamento bovino para guiá-los sem a necessidade de grande intervenção humana.
Outro grande problema resolvido por ela com seu novo design é a perda de animais
afogados nos tanques de água de banho. Temple observou que os bovinos poderiam ser induzidos
a atravessar o tanque apenas modificando o piso de entrada. Assim, perdas de vida
poderiam ser evitadas e a diminuição do número de peões da fazenda.

Por ultimo, Temple, continua suas pesquisas e redefine a linha de produção de um frigorifico
para tornar o processo mais humano e menos cruel com os animais. Utilizando os seus 
conhecimentos adquiridos na reformulação dos currais, ela percebe que pode levar os bovinos
pelo frigorifico sem muita intervenção humana, diminuindo o número de paradas do estabelecimento
causados por problemas gerados na hora do abate.





