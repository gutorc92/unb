\setcounter{secnumdepth}{0}
\section{1) O que é interação emocional?}

O estudo das reaches das pessoas ao interagirem com produtos tecnológicos
como computadores. Essa área de estudo abrange o estudo de como sentimentos
quando interagimos a primeira vez com o sistema e como podemos nos relacionar
com eles. 

\section{2) Cite algumas maneiras para entender como as emoções afetam o
comportamento e como o comportamento afeta as emoções? Cite exemplos?}

As emoções podem ser transmitidas através de comportamentos, por exemplo, uma
pessoa feliz tende a sorrir. O entendimento dessas atitudes ajuda a identificar
as emoções que os usuários sentem ao interagir com o sistema.

\section{3) Explique a razão ou a razões de usar formas expressivas nas 
interfaces. Cite exemplos dessas formas expressivas na história das interfaces.}

A melhoria da interação do computador com os seres humanos sempre foi um 
desafio o uso de formas expressivas de interface podem ajudar a alcançar 
esse objetivo. As formas expressivas são usadas para transmitir estados 
emocionais e provocar certas respostas emocionais, exemplos do dia-a-dia desse
tipo de interface são emojis, sons ícones e animações.

\section{4) Sobre interfaces expressivas, cite exemplos de outras formas para 
transmitir o estado de um sistema para o usuário.}

Como outras formas de interfaces expressivas podem ser citados ícones 
dinâmicos, animações, mensagens faladas, sons indicado ações, entre outras.

\section{5) Quais são as vantagens e/ou desvantagens de utilizar esses tipos 
de detalhes expressivos?}

Uma vantagem de um ícone expressivo é o de gerar um feedback simples para o 
usuário, e que ele entenda emocionalmente. Uma desvantagem é o fato de ter a 
possibilidade de as pessoas acharem esses itens expressivos intrusivos, 
gerando irritação, ou seja, o efeito oposto esperado.

\section{6) Comente sobre os casos de utilização de agentes amigáveis 
(friendly agentes) em interfaces pela Microsoft na década de 1.990. 
Explique se foi um sucesso ou do não sucesso da utilização desses agentes.}
Tanto o Bob (companheiro cachorro), quanto o Clippy foram fracassos na 
tentativa de aplicar agentes amigáveis aos softwares da Microsoft. O primeiro 
nem foi lançado, por ter sido descartado em testes com usuários que acharam 
muito infantil e irritante. O segundo foi lançado, mas foi motivo de muita 
irritação por suas interrupções durante o processo de usar o software em si.

\section{7) Qual a vantagem ou desvantagem de usar um ícone de sinal de estrada
“homens trabalhando” para indicar que um site está em construção?}

A grande desvantagem é a de que o usuário já está frustrado por não ter acesso 
a essa parte do site que está em construção, e pode ser irritar mais ainda por 
causa da brincadeira e trocadilho.

\section{8) Qual os cuidados que deve se ter para enviar uma mensagem de 
erro ao usuário?}

Mensagens atenciosas
Evitar palavras FATAL, ERRO, INVÁLIDO, RUIM, ILEGAL
Evitar grandes números e letras maiúsculas
Avisos sonoros devem ter opção de silenciar
Entre outros

\section{9) Com que os usuários frequentemente se irritam?}

Com mensagens de erro, esperas demoradas na utilização, atualizações 
frequentes e que não dão opção de cancelamento (Windows), 
aparência desagradável, bugs etc.

\section{10) Explique antropomorfismo e zoomorfismo? De exemplos de suas 
aplicação ao design de sistemas.}

Antropomorfismo se refere à utilização de características humanas em algum 
elemento (agentes amigáveis, ou na própria interface, por exemplo).
Zoomorfismo se refere à utilização de características de animais em algum 
elemento. Um exemplo de zoomorfismo é a tentativa de adoção do agente amigável 
Bob da Microsoft, que era um companheiro cachorro.
Um exemplo de antropomorfismo é a tentativa de adoção do agente amigável 
Clippy que se tratava de um clip de papel, com características humanas.
