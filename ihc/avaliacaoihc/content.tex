\setcounter{secnumdepth}{0}
\section{Questão 1}
Avaliar a qualidade de uso é interessante pois garante uma maior qualidade
do produto entregue para o usuário final. Além desse fato, avaliar pode evitar
que erros de iteração possam ser resolvidos antes da entrega final do produto
evitando custos com manutenção. Outro ponto importante é a verificação da
conformidade do produto desenvolvido com o a perspectiva do usuário.

\section{Questão 2}
A avaliação de IHC pode ser auferida em diferentes tipos de dados, podendo
ser citados:
    nominais categorizam os dados em rótulos, por exemplo, etnia, gênero, etc.
    ordinais 
    ordinais a posição da categoria reflete informações sobre ela: Ex: sites
    mais acessados. O primeiro da lista tem mais relevância que os outros.
    qualitativos dados que não se traduzem numericamente. Ex: a opinião sobre
    um determinado componente.
    qualitativos são representados em números. 
    objetivos podem ser auferidos diretamente através de um processo de 
    medição.
    subjetivos devem ser expressos pelo grupo que esta sendo avaliado.


\section{Questão 3}
Os métodos de avaliação de IHC podem ser divididos em três grupos: investigação, 
observação em uso e de inspeção. O primeiro deles é feito através da aplicação
de questionários, entrevistas e estudos de campo e permite avaliar as 
expectativas e comportamentos do usuário. O segundo é feito através da 
observação do usuário no ambiente em que a solução será utilizada permite
identificar problemas reais do usuário em seu dia a dia. O último deles
inspeciona uma solução de IHC para prever seus potenciais problemas, não 
envolve usuários reais e permite tratar experiências de uso potencias.


\section{Questão 4}
O planejamento de uma avaliação de IHC consiste em determinar os objetivos
da avaliação e questões específicas de investigação, o escopo - quais partes
da solução serão avaliados -, os métodos que serão utilizados e o perfil 
do usuários que participarão da avaliação. 

\section{Questão 5}
Um teste piloto visa testar se todos os esquipamentos que serão utilizados na
realização da avaliação de IHC estão funcionando adequadamente.

\section{Questão 6}
Os usuários devem representar os usuários típicos do sistema e que não se 
sintam pressionados a participar da avaliação. Ainda deve-se receber os 
usuários com conforto e privacidade para a realização dos testes.

\section{Questão 7}
Os avaliadores devem apresentar um sumário dos dados coletados que inclua
tabelas e gráficos, uma análise dos dados, uma lista de problemas encontrados
e um planejamento para o reprojeto do sistema. 





